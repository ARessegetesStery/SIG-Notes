\documentclass[10pt]{article}
\usepackage{../../survey}

\addbibresource{bib.bib}

\begin{document}

\Makepagesectionhead{Walking On Stars with Boundary Conditions}{ARessegetes Stery}

\tableofcontents
\clearpage

\section{Preliminaries}

\textstart
To get familiar with the context, first present the definition of the various boundary conditions:

\begin{definition}
    Given an (ordinary/partial) differential equation with domain $\Omega$,
    \begin{itemize}
        \item A \textbf{Dirichlet boundary condition} fixes the value of solution on the boundary of domain. 
        \item A \textbf{Neumann boundary condition} fixes the derivative (normal) applied at the boundary of the domain.
        \item A \textbf{Robin boundary condition} is a weighed combination of the previous two. Explicitly, for given \emph{functions} $a, b, g$ defined on $\partial \Omega$ the associated Robin boundary condition is (for target function $f$)
        \[
            a\ f + b\ \partial_n f = g \qquad \text{on $\partial \Omega$}
        \]
        where $\partial_n(\cdot)$ denotes the normal derivative.
    \end{itemize}
\end{definition}

\textstart
The followings are a collection some purely Mathematical definitions for formal description of objects introduced. They are not necessarily essential for understanding the objects, and serves as a reminder merely. You are welcome (and suggested) to skip this part, and refer back to these definitions when encountering anything unfamiliar.

\begin{definition}[$\sigma$-algebra]
    Given a set $X$ with $\P(X)$ its power set, a subset $\Sigma \subseteq P(X)$ is a \textbf{$\sigma$-algebra} if it satisfies
    \begin{enumerate}[label=\arabic*)]
        \item $X \in \Sigma$.
        \item $\Sigma$ is closed under complement.
        \item $\Sigma$ is closed under countable unions.
    \end{enumerate}
\end{definition}
\nogap
\begin{remark}
    By applying De Morgan's Law directly, $\sigma$-algebras are also closed under countable intersections.
\end{remark}

\begin{definition}[Borel (Measurable) Space]
    A \textbf{Borel Space}, (or Measurable Space), is a tuple $(X, \mathcal{F})$ where $\mathcal{F}$ is a $\sigma$-algebra on $X$.
\end{definition}
\nogap
\begin{remark}
    This needs to be distinguished from the \emph{measure space}: no measure is required for a measurable space. The ``measurable'' here refers to the sets in $\mathcal{F}$ are ``measured'', or considered, in $\P(X)$.
\end{remark}

\begin{definition}[Stochastic Process]
    A \textbf{stochastic process} on a probability space $(\Omega, \mathcal{F}, \Pr)$ with a measurable space $(S, \Sigma)$ and index set $T$ (often time, subset of $\R$) is a collection of $S$-valued random variables with evaluations $\{ X(t) \mid t \in T \}$.
\end{definition}

\section{Solving Dirichlet Problems by ``Walk On Spheres'' \cite{WalkOnSphere}}

Due to the complexity of algebraic manipulation/approximation, some advanced results will be used as black box, and will be marked with {\color{darkpurple} purple}. Please refer to the original paper for details. Also, results with detailed proofs in the original paper will be reasoned with the ``intuitions''; and important results presented without proof in the paper will be explained in detail. 

\begin{definition}[Harmonic Function]
    A twice continuously differentiable $\R$-valued function $f: U \to \R$ is \textbf{harmonic} if $\nabla^2 f \equiv 0$ on $U$.
\end{definition}
\nogap
\begin{definition}[Dirichlet Problem]
    Given a domain (for simplicity assume that this is simply-connected) $D$ with boundary $\partial D$, a \textbf{Dirichlet problem} associated to a function $f$ continuously defined on $\partial D$ is to find the function $u$ defined on $R$ s.t. $u$ and $f$ coincide on $\partial D$.
\end{definition}

\subsection{Discretization of Brownian Motion}

\begin{definition}[Brownian Motion]
    A \textbf{$\R^d$-valued Brownian motion} starting at $x \in \R^d$ is a stochastic process $\{ B(t) \mid t \in T := \R_{\geq 0} \}$ satisfying the following properties:
    \begin{enumerate}[label=\arabic*)]
        \item \emph{Anchor:} $B(0) = x$.
        \item \emph{Independent incrementals:} for any increasing sequence $(t_n)_{n \in \Z_{\geq 0}}$ on $T$, $\{ B(t_{i+1}) - B(t_i) \mid i \in \Z_{\geq 0} \}$ are independent random variables.
        \item \emph{Normality in each step:} For all $t \geq 0, h > 0$, the incremental $B(t + h) - B(t)$ follows a normal distribution $N(0, h)$.
        \item \emph{Continuity:} The function $t \mapsto B(t)$ is almost surely (i.e., has probability 1 of being) continuous.
    \end{enumerate}
\end{definition}
\nogap
\begin{remark}
    Property 4) in the definition actually loosens the definition; but the discontinuity does not interfere with any numerical treatment, as it happens with probability 0.
\end{remark}
\nogap
\begin{notation}
For simplicity, the evaluation of Brownian motion $B$ will be denoted as $X(t, \omega)$ with $\omega \in \R$ a random variable specifying the exact value of $B(t)$. That is, for $X(t, \omega) \in \R^N$. 
\end{notation}

\textstart
In a domain $D$, given an arbitrary Brownian motion $B$, it can be discretized into an infinite sequence of points (associated to the corresponding time index $t$), where the distribution (w.r.t. $t$) is preserved:

\begin{definition}[Maximum Sphere]
    Let $D$ be an arbitrary domain, and $x \in D$ be an arbitrary point. Then the \textbf{maximum sphere} centered at $x$ in $D$, denoted $\bm{K}_D(x)$, is the sphere
    \[
        S(x, \inf_{x_0 \in \partial D} \{ \norm{x - x_0} \})
    \]
    Its boundary (surface of the sphere) is denoted as $\overline{\bm{K}}_D(x)$.
\end{definition}
\nogap
\begin{definition}[Successive Intersections]
    Given a Brownian motion $B$ with arbitrary ambient domain $D$, its \textbf{successive intersections} w.r.t. $D$ is a (possibly, and often, infinite) sequence $P_n(B(0), \bm{K}_D(P_{n-1}), \omega)$ defined as follows:
    \begin{enumerate}
        \item $P_0(B(0), -, \omega) := B(0)$.
        \item $P_{n}(B(0), \overline{\bm{K}}(P_{n-1}), \omega)$ is the point where $X(t, \omega)$ for the first time intersects with $\overline{\bm{K}}_D(P_{n-1})$. 
    \end{enumerate}
\end{definition}

\textstart
Notice that by the randomness defined in the Brownian motion, the distribution of $P_n$ on $\overline{\bm{K}}_D(P_{n-1})$ is uniform; this is also a Markov chain since the local direction $P_n - P_{n-1}$ is uniformly random, and is independent over $n$ (by requirement 2) in the definition of Brownian motion). Therefore this can be ``perfectly simulated'' with the following process:

\begin{definition}[Spherical Process]
    Given a domain $D$ with boundary $\partial D$ and an arbitrary point $x \in D$, the \textbf{spherical process} defines a sequence of points $P_n(x, \phi)$ where $\phi$ is the random variable, as follows:
    \begin{enumerate}
        \item $P_0(x, \phi) := x$.
        \item For all $i \in \Z_{\geq 1}$, designate $P_i$ uniformly randomly from $\overline{\bm{K}}_D (P_{i-1})$. 
    \end{enumerate}
\end{definition}

It is then clear, that given a Brownian motion with random variable $\omega = \omega_0$ it corresponds uniquely to a spherical process with random variable $\phi = \phi(\omega_0)$; and vice versa. This allows application of the results proved for Brownian motion to obtain the solution of the Dirichlet problem; and use Monte Carlo methods on the spherical process for numerical quadrature.

\subsection{Obtaining Solution from Spherical Process}

\textstart
The solution of the Dirichlet problem associated to function $f$ is computed directly using the following theorems, whose proofs are cited from \cite{Kakutani1944}. Notice that the ``measure'' in the integration, the probability implicitly uses the ``sphere process'' introduced above.

\subsection{Convergence Analysis}

\textstart
This section dedicates to prove the following theorem, with some closely related facts:

\begin{theorem}
    Let $N$ be the dimension s.t. the problem domain $\Omega \subseteq \R^N$. Then the steps required for convergence of the method is linear in $N$. 
\end{theorem}

\section{Boundary Value Caching for \emph{WoS} \cite{BVCWoS}}

\section{Walking on Stars \cite{WalkOnStars}}

\section{Extending \emph{WoSt} to Robin Boundary Conditions\cite{WalkinRobin}}

\printbibliography
\end{document}