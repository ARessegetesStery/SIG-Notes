\documentclass[10pt]{article}
\usepackage{../../survey}

\addbibresource{bib.bib}

\begin{document}

\Makepagesectionhead{Walking On Stars with Boundary Conditions}{ARessegetes Stery}

\tableofcontents
\clearpage

\section{Preliminaries}

\textstart
To get familiar with the context, first present the definition of the various boundary conditions:

\begin{definition}
    Given an (ordinary/partial) differential equation with domain $\Omega$,
    \begin{itemize}
        \item A \textbf{Dirichlet boundary condition} fixes the value of solution on the boundary of domain. 
        \item A \textbf{Neumann boundary condition} fixes the derivative (normal) applied at the boundary of the domain.
        \item A \textbf{Robin boundary condition} is a weighed combination of the previous two. Explicitly, for given \emph{functions} $a, b, g$ defined on $\partial \Omega$ the associated Robin boundary condition is (for target function $f$)
        \[
            a\ f + b\ \partial_n f = g \qquad \text{on $\partial \Omega$}
        \]
        where $\partial_n(\cdot)$ denotes the normal derivative.
    \end{itemize}
\end{definition}

\textstart
The followings are a collection some purely Mathematical definitions for formal description of objects introduced. They are not necessarily essential for understanding the objects, and serves as a reminder merely. You are welcome (and suggested) to skip this part, and refer back to these definitions when encountering anything unfamiliar.

\begin{definition}[$\sigma$-algebra]
    Given a set $X$ with $\P(X)$ its power set, a subset $\Sigma \subseteq P(X)$ is a \textbf{$\sigma$-algebra} if it satisfies
    \begin{enumerate}[label=\arabic*)]
        \item $X \in \Sigma$.
        \item $\Sigma$ is closed under complement.
        \item $\Sigma$ is closed under countable unions.
    \end{enumerate}
\end{definition}
\nogap
\begin{remark}
    By applying De Morgan's Law directly, $\sigma$-algebras are also closed under countable intersections.
\end{remark}

\begin{definition}[Borel (Measurable) Space]
    A \textbf{Borel space}, (or \textbf{measurable space}), is a tuple $(X, \mathcal{F})$ where $\mathcal{F}$ is a $\sigma$-algebra on $X$.
\end{definition}
\nogap
\begin{remark}
    This needs to be distinguished from the \emph{measure space}: no measure is required for a measurable space. The ``measurable'' here refers to the sets in $\mathcal{F}$ are ``measured'', or considered, in $\P(X)$.
\end{remark}

\begin{definition}[Stochastic Process]
    A \textbf{stochastic process} on a probability space $(\Omega, \mathcal{F}, \Pr)$ with a measurable space $(S, \Sigma)$ and index set $T$ (often time, subset of $\R$) is a collection of $S$-valued random variables with evaluations $\{ X(t) \mid t \in T \}$.
\end{definition}

\section{Solving Dirichlet Problems by the ``Spherical Process'' \cite{WalkOnSphere}}

Due to the complexity of algebraic manipulation/approximation, some advanced results will be used as black box, and will be marked with {\color{darkpurple} purple}. Please refer to the original paper for details. Also, results with detailed proofs in the original paper will be reasoned with the ``intuitions''; and important results presented without proof in the paper will be explained in detail. 

\begin{definition}[Harmonic Function]\label{def: harmonic function}
    A twice continuously differentiable $\R$-valued function $f: U \to \R$ is \textbf{harmonic} if $\nabla^2 f \equiv 0$ on $U$.
\end{definition}
\nogap
\begin{definition}[Dirichlet Problem]
    Given a domain (for simplicity assume that this is simply-connected) $D$ with boundary $\partial D$, a \textbf{Dirichlet problem} associated to a function $f$ continuously defined on $\partial D$ is to find the function $u$ defined on $R$ s.t. $u$ and $f$ coincide on $\partial D$.
\end{definition}

\subsection{Discretization of Brownian Motion}

\begin{definition}[Brownian Motion]
    A \textbf{$\R^d$-valued Brownian motion} starting at $x \in \R^d$ is a stochastic process $\{ B(t) \mid t \in T := \R_{\geq 0} \}$ satisfying the following properties:
    \begin{enumerate}[label=\arabic*)]
        \item \emph{Anchor:} $B(0) = x$.
        \item \emph{Independent incrementals:} for any increasing sequence $(t_n)_{n \in \Z_{\geq 0}}$ on $T$, $\{ B(t_{i+1}) - B(t_i) \mid i \in \Z_{\geq 0} \}$ are independent random variables.
        \item \emph{Normality in each step:} For all $t \geq 0, h > 0$, the incremental $B(t + h) - B(t)$ follows a normal distribution $N(0, h)$.
        \item \emph{Continuity:} The function $t \mapsto B(t)$ is almost surely (i.e., has probability 1 of being) continuous.
    \end{enumerate}
\end{definition}
\nogap
\begin{remark}
    Property 4) in the definition actually loosens the definition; but the discontinuity does not interfere with any numerical treatment, as it happens with probability 0.
\end{remark}
\nogap
\begin{notation}
For simplicity, the evaluation of Brownian motion $B$ will be denoted as $X(t, \omega)$ with $\omega \in \R$ a random variable specifying the exact value of $B(t)$. That is, for $X(t, \omega) \in \R^N$. 
\end{notation}

\textstart
In a domain $D$, given an arbitrary Brownian motion $B$, it can be discretized into an infinite sequence of points (associated to the corresponding time index $t$), where the distribution (w.r.t. $t$) is preserved:

\begin{definition}[Maximum Sphere]
    Let $D$ be an arbitrary domain, and $x \in D$ be an arbitrary point. Then the \textbf{maximum sphere} centered at $x$ in $D$, denoted $\bm{K}_D(x)$, is the sphere
    \[
        S(x, \inf_{x_0 \in \partial D} \{ \norm{x - x_0} \})
    \]
    Its boundary (surface of the sphere) is denoted as $\overline{\bm{K}}_D(x)$.
\end{definition}
\nogap
\begin{definition}[Successive Intersections]
    Given a Brownian motion $B$ with arbitrary ambient domain $D$, its \textbf{successive intersections} w.r.t. $D$ is a (possibly, and often, infinite) sequence $P_n(B(0), \bm{K}_D(P_{n-1}), \omega)$ defined as follows:
    \begin{enumerate}
        \item $P_0(B(0), -, \omega) := B(0)$.
        \item $P_{n}(B(0), \overline{\bm{K}}(P_{n-1}), \omega)$ is the point where $X(t, \omega)$ for the first time intersects with $\overline{\bm{K}}_D(P_{n-1})$. 
    \end{enumerate}
\end{definition}

\textstart
Notice that by the randomness defined in the Brownian motion, the distribution of $P_n$ on $\overline{\bm{K}}_D(P_{n-1})$ is uniform; this is also a Markov chain since the local direction $P_n - P_{n-1}$ is uniformly random, and is independent over $n$ (by requirement 2) in the definition of Brownian motion). Therefore this can be ``perfectly simulated'' with the following process:

\begin{definition}[Spherical Process]
    Given a domain $D$ with boundary $\partial D$ and an arbitrary point $x \in D$, the \textbf{spherical process} defines a sequence of points $P_n(x, \phi)$ where $\phi$ is the random variable, as follows:
    \begin{enumerate}
        \item $P_0(x, \phi) := x$.
        \item For all $i \in \Z_{\geq 1}$, designate $P_i$ uniformly randomly from $\overline{\bm{K}}_D (P_{i-1})$. 
    \end{enumerate}
\end{definition}

It is then clear, that given a Brownian motion with random variable $\omega = \omega_0$ it corresponds uniquely to a spherical process with random variable $\phi = \phi(\omega_0)$; and vice versa. This allows application of the results proved for Brownian motion to obtain the solution of the Dirichlet problem; and use Monte Carlo methods on the spherical process for numerical quadrature.

\subsection{Obtaining Solution from the Spherical Process}

\textstart
The solution of the Dirichlet problem associated to function $f$ is computed directly using the following result. A complete context of the result can be found in \cite{Kakutani1944}. Notice that the ``measure'' in the integration involves the aforementioned Brownian motion. \hyperlink{https://willierushrush.github.io/posts/2021/08/dirichlet-problem}{This blog} provides a good excursion into the topic. 

\begin{proposition}[Kakutani's Principle \cite{KakutaniProof}]
    Given a Dirichlet problem defined on a domain $D$ with boundary values $f(\zeta)$ for $\zeta \in \partial D$, for any $x_0 \in D$, the solution $u$ of the Dirichlet problem evaluates to
    \[
        u(x_0) = \int_{\partial D} \Pr(x_0, \zeta, D) f(\zeta) \d \zeta \qquad \forall x_0 \in D
    \]
    where $\Pr(x_0, \zeta, D)$ is the probability density function of a Brownian motion starting at $x_0$ intersects for the first time with $\partial D$ at $\zeta$.
\end{proposition}

\begin{proof}
    In this proof we use without validation the result that every Dirichlet problem admits a unique solution (existence and uniqueness); and a function being harmonic is equivalent to conforming to the ``mean value property''. Details can be found in \hyperref[appendix A]{Appendix A}.

    \begin{paraindent}
        \begin{definition}[Mean Value Property]\label{def: mean value property}
            A function $f$ with support $D \subset \R^N$ satisfies the \textbf{mean value property} if for all $r \in \R, x \in D$ s.t. $\mathcal{S}(x, r)$ (the radius-$r$ $N$-dimensional ball centered at $x$) is contained in $D$, the following equality is satisfied
            \[
                f(x) = \frac{1}{\abs{\partial \mathcal{S}(x, r)}}\int_{\partial \mathcal{S}(x, r)} f(\zeta) \d \zeta
            \]
            or equivalently, expressed using radians,
            \[
                f(x) = \frac{1}{2 \pi} \int_0^{2\pi} f(x + r e^{i \theta}) \d \theta 
            \]
        \end{definition}
    \end{paraindent}

    To prove the proposition given the assumptions on the function, we only need to further verify the following two properties:
    \begin{enumerate}
        \item The function $u$ is a harmonic function.
        \item The function $u$ coincides with $f$ on $\partial D$.
    \end{enumerate}

    For the first property, prove using the equivalence with the mean value property: as we have mentioned, the Brownian motion is ``locally memory-less'', i.e. formally
    \[
        \E[B(t) \mid B(0) = x, B(t_0) = y] = \E[B(t - t_0) \mid B(0) = y]
    \]
    Use this transformation to verify the mean value property. Let $U = \mathcal{S}(x_0, r)$ be a ball centered at $x_0$, with radius $r$, and completely contained in $D$. Denote the point where a Brownian motion starting at $X(0, \omega) = x_0$ meet $\partial D$ ($\partial U$) for the first time by $P(x_0, \omega, D)$ ($P(x_0, \omega, U)$); and the corresponding time of encounter $T(x_0, \omega, D)$ ($T(x_0, \omega, U)$). By conditional probability with an argument of partition of unity we have
    \[
        u(x_0) = \int_{\partial D} \Pr(x_0, \zeta, D) f(\zeta) \d \zeta = \E[u(P(x_0, \omega, D))] = \frac{1}{\abs{\partial U}} \int_{\partial U} \E[f(P(x_0, \omega, D)) \mid P(x_0, \omega, U) = y]\ \d y
    \]
    Using the memory-less property, we have
    \[
        \frac{1}{\abs{\partial U}} \int_{\partial U} \E[f(P(x_0, \omega, D)) \mid P(x_0, \omega, U) = y]\ \d y = \frac{1}{\abs{\partial U}} \int_{\partial U} \E[ f(P(y, \omega, D)) ]\ \d y = \frac{1}{\abs{\partial U}} \int_{\partial U} u(y) \d y
    \]
    where the last equality results from the definition of $u$. 

    Now verify that $u$ has boundary condition $f$ on $\partial D$, that is, equivalently, for all $\zeta \in \partial D$,
    \[
        \lim_{x \to \zeta, x \in D} \E[f(P(x, \omega, D))] = f(\zeta)
        \quad \Longleftrightarrow \quad
        \lim_{x \to \zeta, x \in D} \E[f(P(x, \omega, D)) - f(\zeta)] = 0
    \]
    since the expectation of a deterministic function is itself. Consider in turn a shrinking domain $S_r := \mathcal{S}(\zeta, r)$ with $r \to 0$. Adopting the same notation as above, we have
    \begin{align*}
        & \lim_{x \to \zeta, x \in D} \E[f(P(x, \omega, D)) - f(\zeta)] \\
        = & \lim_{x \to \zeta, x \in D} \E[f(P(x, \omega, D)) - f(\zeta) \mid T(x, \omega, S_r) < T(x, \omega, D)] + \\
        & \qquad \qquad \lim_{x \to \zeta, x \in D} \E[f(P(x, \omega, D)) - f(\zeta) \mid T(x, \omega, S_r) \geq T(x, \omega, D)] \\
        \leq &\ \underbrace{\Pr[T(x, \omega, S_r) < T(x, \omega, D)]}_{(\ast)} \cdot (\abs{f(P(x, \omega, D))} + f(\zeta)) + \underbrace{\sup_{\eta \in S \cap \partial D} \abs{f(\eta) - f(\zeta)}}_{(\ast\ast)} \cdot \Pr[T(x, \omega, S_r) \geq T(x, \omega, D)]
    \end{align*}
    Now consider $x \in S_r$ with $r \to 0$. Then for $r \to 0$ with $x \to \zeta$, $(\ast) \to 0$; and for $r \to 0$ by continuity of $f$ $(\ast\ast) \to 0$. This gives $\lim_{x \to \zeta, x \in D} \E[f(P(x, \omega, D)) - f(\zeta)] = 0$; and thus the coincidence between $u$ and $f$ on $\partial D$.

    {\color{red} TODO answer in mathexchange}
\end{proof}

\begin{remark}
    The probability function $\Pr(x_0, \zeta, D) \d \zeta =: \d \omega$ defines a measure, and is called the \underline{harmonic measure}.
\end{remark}

\textstart
Numerically, the quadrature can be approximated via applying Monte Carlo analysis on the spherical process.

\subsection{Convergence Analysis}

\textstart
First we need to show that the spherical process indeed (almost always, i.e. with probability 1) converges.

\begin{proposition}[{\color{darkpurple} \cite{WalkOnSphere} Theorem 3.6}]
    The sequence $(P_i(x_0 [:= B(0)], -, D))_{i \in \Z_{\geq 0}}$ obtained by the spherical process converges, with probability 1, to a point on $\partial D$, and coincides with the first intersection between $B(t)$ and $\partial D$.
\end{proposition}

\begin{proof}[Intuition]
    By the analogy between the spherical process and random walk, $B(t)$ intersects $\partial D$ at some finite time $t_0$ for the first time with probability 1. Then:
    \begin{enumerate}
        \item Consider the time sequence $T_i$ taking values in $\R$ specifying the time of Brownian motion corresponding to the spherical process fixed by sequence $(P_i)_{i \in \Z_{\geq 0}}$ coming across the point $P_i$. It is strictly increasing, and bounded above by $t_0$, which implies that it has a limit, denoted $\bar{t_0}$. Therefore \emph{$(P_i)_{i \in \Z_{\geq 0}}$ converges}.
        \item Consider $\bar{P} = B(\bar{t_0})$ which by definition of $t_0$ must be either in interior of $D$ or on $\partial D$. If $\bar{P}$ is in the interior, then $\bar{K}(\bar{P})$ has nonzero diameter. Specifically, choosing $\varepsilon$ small enough gives, for any $n$ large enough s.t. $P_n \to \bar{P}$, $\norm{P_n - P_{n+1}} > \varepsilon$ with probability 1, where $P_{n+1}$ is the resulting point of iterating again using the spherical process. Therefore \emph{$(P_i)_{i \in \Z_{\geq 0}}$ converges to the interior of $D$}.
        \item The spherical process cannot produce points outside $D$; and by the minimality of $t_0$ this gives the coincidence between $\lim_{n \to \infty} P_n$ and $B(t_0)$. 
    \end{enumerate}
\end{proof}

\textstart
Now we show the order of convergence:

\begin{theorem}[{\color{darkpurple} \cite{WalkOnSphere} Chapter 6}]
    Let $N$ be the dimension s.t. the problem domain $\Omega \subseteq \R^N$. Then the steps required for convergence of the method is linear in $N$ if $D$ is convex. 
\end{theorem}

\begin{proof}[Sketch of Proof]
    First we show this for a simplified scenario. Consider the case where $\partial D$ is a hyperplane in $\R^N$, i.e. a plane identifiable with $\R^{N - 1}$. Define the coordinate system for $\R^N$ as $(y, \cdots)$ s.t. for $P_i = (y_i, \cdots)$, the distance from $P_i$ to $\partial D$ is $y_i$. Now for a starting point of the spherical process $P_0 = (y_0, \cdots)$ we consider the sequence $(y_i)_{i \in \Z_{\geq 0}}$:

    \begin{paraindent}
        \begin{lemma}[{\color{darkpurple} \cite{WalkOnSphere} Theorem 6.1, 6.2}]
            Under the above setup, we have the asymptotic result for $N \to \infty$
            \[
                \E\left( \left. \log \frac{y_{i+1}}{y_i} \right| N \right) \sim -\frac{1}{2N}
            \]
        \end{lemma}

        \begin{proof}[Sketch of Proof]
            First notice that the $y_{i}$s are related via $\frac{y_{i+1}}{y_i} = 1 - \cos \theta_{i+1}$, where $\theta_{i+1}$ is the angle formed by the normal of $\partial D$ and $(P_{i+1} - P_i)$. Integrating the probability on the sphere gives
            \[
                \E\left( \left. \log \frac{y_{i+1}}{y_i} \right| N \right) = \frac{\int_{0}^{\pi} \log (1 - \cos \theta) \sin^{N-2} \theta \d \theta}{\int_0^{\pi} \sin^{N-2} \theta \d \theta}
            \]
            Manipulating the expression above gives the asymptotic. Refer to the original paper for details.
        \end{proof}
    \end{paraindent}

    Then the expected steps of convergence is 
    \[
        \frac{1}{ \abs{\left( \left. \log \frac{y_{i+1}}{y_i} \right| N \right)}} \sim N = O(N) 
    \]
    Now we seek to extend this to general convex sets. For each $P_i$ consider the point $\zeta_i \in \partial D$ where the minimum distance $\text{radius}(P_i)$ is realized. Let $\partial D_i$ be the tangent plane of $\partial D$ at $\zeta_i$, with the corresponding distance coordinate $y_{(\cdot)}^{(i)}$. Notice that $y_{i+1}^{(i)} \geq y_{i+1}^{(i+1)}$ by minimality. Therefore $y_{(\cdot)}^{(0)}$, which converges to 0 in $O(N)$ steps, gives an upper bound for number of steps required for convergence. This implies the actual convergence order is also $O(N)$. 
\end{proof}

\begin{remark}
    Notice that there are two major deficiencies for \emph{Walking on Sphere} method:
    \begin{itemize}
        \item The convergence order can only be proven for convex domains, whose proof relies heavily on the property that the whole domain lies on one side of the hyperplane.
        \item The continuities of both boundary constraint $f$ and target function $u$ are not utilized.
    \end{itemize}
    These are the direction of improvements in the subsequent papers.
\end{remark}

\section{Walking on Stars \cite{WalkOnStars}}

\section{Boundary Value Caching for \emph{WoS} \cite{BVCWoS}}

\section{Extending \emph{WoSt} to Robin Boundary Conditions\cite{WalkinRobin}}

\printbibliography

\section{Appendix A - Existence of Solution to the Dirichlet Problem}\label{appendix A}

First we prove the equivalence:

\begin{proposition}
    For a twice-differentiable function $f$ on a domain $D$, the following two statements are equivalent:
    \begin{enumerate}
        \item $f$ is \hyperref[def: harmonic function]{harmonic} on $D$.
        \item $f$ satisfies the \hyperref[def: mean value property]{mean value property}.
    \end{enumerate}
\end{proposition}

\begin{proof}
    {\color{red} TODO}
\end{proof}

Now for the proof of Kakutani's Principle, we only need to show that every Dirichlet problem admits a unique solution:

\begin{proposition}
    For any regular domain $D$ with a continuous function $f$ defined on $\partial D$, the Dirichlet problem
    \[
        \begin{cases}
            u(x) = f(x), \ \forall x \in \partial D \\
            \Delta u(x) = 0, \ \forall x \in D
        \end{cases}
    \]
    admits a unique solution
\end{proposition}

\begin{proof}
    {\color{red} TODO}
\end{proof}

\end{document}