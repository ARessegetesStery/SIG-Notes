\documentclass{beamer}

\usepackage{../../presentation}

\title{Walkin' Robin \\[1pt] {\Large Walking on Stars with Robin Boundary Conditions}}
\author{ARessegetes Stery}
\date{July 11, 2024}

\begin{document}

\maketitle

\colortheme{lightgreen}

\section{Random ``Walk On Sphere''}

\begin{frame}{Various Boundary Conditions}

	\begin{definition}
		Given an (ordinary/partial) differential equation with domain $\Omega$,
		\begin{itemize}
			\item A \textbf{Dirichlet boundary condition} fixes the value of solution on the boundary of domain. 
			\item A \textbf{Neumann boundary condition} fixes the derivative (normal) applied at the boundary of the domain.
			\item A \textbf{Robin boundary condition} is a weighed combination of the previous two. Explicitly, for given \emph{functions} $a, b, g$ defined on $\partial \Omega$ the associated Robin boundary condition is (for target function $f$)
			\[
				a\ f + b\ \partial_n f = g \qquad \text{on $\partial \Omega$}
			\]
			where $\partial_n(\cdot)$ denotes the normal derivative.
		\end{itemize}
	\end{definition}
	\horzline

\end{frame}

\colortheme{lightblue}

\section{Improvements in ``Walking On Stars''}

\begin{frame}
    
\end{frame}

\colortheme{lightpurple}

\section{Theoretical Background}

\begin{frame}{Automorphisms}

\end{frame}

\colortheme{grey}

\section{Applied Approaches}

\begin{frame}{Sample Frame}

	Applied stuff.

\end{frame}

\colortheme{default}

\thankframe

\end{document}
